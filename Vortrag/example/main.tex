% !TeX spellcheck = de_DE
\documentclass[ngerman,hyperref={pdfpagelabels=false}]{beamer}

% -----------------------------------------------------------------------------

\graphicspath{{images/}}

% -----------------------------------------------------------------------------

\usetheme{KIT}

\setbeamercovered{transparent}
%\setbeamertemplate{enumerate items}[ball]

\newenvironment<>{KITtestblock}[2][]
{\begin{KITcolblock}<#1>{#2}{KITblack15}{KITblack50}}
{\end{KITcolblock}}

\usepackage[ngerman,english]{babel}
\usepackage[utf8]{inputenc}
\usepackage[TS1,T1]{fontenc}
\usepackage{array}
\usepackage{multicol}
\usepackage[absolute,overlay]{textpos}
\usepackage{beamerKITdefs}
\usepackage{amsfonts}


\pdfpageattr {/Group << /S /Transparency /I true /CS /DeviceRGB>>}	%required to prevent color shifting withd transparent images


\title{A Linear-Time Algorithm For Finding Tree-Decompositions Of Small Treewidth}
\subtitle{Autor: Hans L. Bodlaender -- \textit{h.l.bodlaender@uu.nl} \\
	Vortrag: Maximilian F. Göckel -- \textit{uzkns@student.kit.edu}}

\author[Maximilian Göckel]{Maximilian F. Göckel}
\institute{Institut für Theoretische Informatik - Proseminar Algorithmen für NP-schwere Probleme}

\TitleImage[width=\titleimagewd,height=\titleimageht]{titel}

\KITinstitute{Institut f\"ur Theoretische Informatik}
\KITfaculty{Proseminar NP-schwere Probleme}

% -----------------------------------------------------------------------------

\begin{document}
\setlength\textheight{7cm} %required for correct vertical alignment, if [t] is not used as documentclass parameter


% title frame
\begin{frame}
  \maketitle
\end{frame}



%Voraussetzungen
\begin{frame}
\frametitle{Tree-decomposition}
\framesubtitle{Definition}
\ \\ 
Eine Baumzerteilung eines Graphen $G=(V,E)$ ist ein Tupel $(X, T)$ wo $T =(I, F)$ ein Baum ist und $X = \{ X_i | i \in I \}$ eine Familie von Teilmengen von $V$ wobei jedes $X_i$ einen Knoten in $T$ darstellt. Es gilt: \\
\ \\
\begin{enumerate}
	\item $\bigcup\limits_{i \in I} X_i = V$
	\item $\forall (v,w) \in E: \exists i \in I: v,w \in X_i$
	\item $\forall w \in X_i, X_j : $ Jedes $X_k$ im Pfad zwischen $X_i, X_j$ enthält $w$
\end{enumerate}
\end{frame}


%Beispiel
\begin{frame}
\frametitle{Tree-decomposition}
\framesubtitle{Veranschaulichung}

\begin{figure}[htbp] 
	\centering
	\includegraphics[width=0.7\textwidth]{tree_2.png}
\end{figure}

\end{frame}


%\begin{frame}
%\frametitle{Tree-decomposition: Beispiel}
%
%\begin{figure}[htbp] 
%	\centering
%	\includegraphics[width=0.7\textwidth]{tree_1.png}
%	\caption{2}
%	\label{fig:Bild1}
%\end{figure}
%
%\end{frame}


%treewidth
\begin{frame}
\frametitle{Treewidth}
\framesubtitle{Definition}

Jede Baumzerteilung hat eine "Baumweite" (treewidth). \\

\begin{itemize}
	\item Baumweite einer Zerteilung: ${\max}( {|X_i|}_{i \in I} - 1)$ ("Zerteilungsweite")
	\item Baumweite eines Graphen: Minimale Zerteilungsweite aller Zerteilungen
\end{itemize}

Eine Baumzerteilung der Weite max. $k$ heißt auch $k$-Baumzerteilung oder $k$-Zerteilung.

\end{frame}


\begin{frame}
\frametitle{k-Trees}
\framesubtitle{Definition}
Ein $k$-Tree $G=/V,E)$ ist ... \\
\begin{itemize}
	\item induktiv konstruiert aus einem $k$-Tree mit $|V|-1$ Knoten indem man einen Knoten zu einer Clique max. $k$ hinzufügt oder
	\item ein Graph der eine Clique max. $k$ ist
\end{itemize}
\ \\
Ein $k$-Tree ist der maximale Graph mit Baumweite $k$. \\
\end{frame}


%partial k-trees

\begin{frame}
\frametitle{Partial k-Trees}
\framesubtitle{Definition}

Graph $G=(V,E)$ ist partieller $k$-Tree $\Leftrightarrow$ \\

\begin{itemize}
	\item $G$ ist Teilgraph eines $k$-Trees oder
	\item $G$ hat Baumweite max. $k$ %+Beweis?
\end{itemize}

\end{frame}


%Anwendungen Baumzerteilungen
\begin{frame}
\frametitle{Baumzerteilung}
\framesubtitle{Anwendungen}

\begin{itemize}
	\item Maximum-Weight Independent Set in Linearzeit lösbar
	\item Hohe Baumweite $\Leftrightarrow$ Hohe Komplexität in der Systemanalyse
	\item Erkennungsalgorithmen von Graphen
	\begin{itemize}
		\item Graphen mit geb. Pfadweite
		\item Partielle $k$-Trees
	\end{itemize}
\end{itemize}
\end{frame}


\begin{frame}
\frametitle{Algorithmus}
\framesubtitle{Allgemeines}

Eingabe: Graph $G=(V,E)$ und Konstante $k \in \mathbb{N}$. \\
Der Graph wird als Adjazenzliste übergeben.
\ \\
Ausgabe in $O(n)$:
\begin{itemize}
	\item "Baumweite von $G$ ist größer als $k$" oder
	\item "Baumweite von $G$ ist maximal $k$"
	\begin{itemize}
		\item Baumzerteilung von $G$ mit Baumweite $k$
	\end{itemize}
\end{itemize}
\ \\
\ \\
Ist $k$ Teil der Eingabe so ist das Problem NP-schwer und sogar NP-vollständig.
\end{frame}


\begin{frame}
\frametitle{Knotentypen}
\framesubtitle{Simplizial, freundlich, low- und highdegree}
Ein Knoten $v$ ist ... \\
\begin{itemize}
	\item ... \emph{von niedrigem Grad} wenn deg($v$) $\leq d$
	\begin{itemize}
		\item $d := 2k^3 \cdot (k+1) \cdot (4k^2 +12k + 16)$
		\item Analog: Hoher Grad $\Leftrightarrow$ deg($v$) $> d$
		\item Auch "low-deg.-" und "high-deg.-Knoten" genannt
	\end{itemize}
	\item ... \emph{Friendly} wenn er low-deg. und adjazent zu einem weiteren low-deg.-Knoten ist
	\item ... \emph{Simplizial} wenn alle Nachbarn eien Clique formen
\end{itemize}
\end{frame}


\begin{frame}
\frametitle{Maximal Matching $M \subseteq E$}

$M \subseteq E$ ist Matching in $G=(V,E) \Leftrightarrow$ Keine 2 Kanten aus $M$ haben gemeinsamen Endknoten \\
$M \subseteq E$ ist Maximal Matching wenn keine Kante mehr zu $M$ hinzugefügt werden kann, sodass $M$ Matching bleibt. \\
\ \\
Ein Maximal Matching kann in $O(|V|)$ gefunden werden, wenn die Baumweite durch ein $k$ beschränkt ist. \\ %LEMMA 2.3.
\begin{figure}
	\includegraphics[height=0.4\textheight]{images/maximum_matching.png}
\end{figure}

\end{frame}


\begin{frame}


\begin{KITinfoblock}{Lemma 4.2.}
	$G$ hat Baumweite max. $k \Rightarrow$ Es gilt mindestens:
	\begin{enumerate}
		\item $G$ hat min. $\frac{|V|}{4k^2+12k+16} =: \lambda$ Friendly-Knoten, oder
		\item Der Improved-Graph von $G$ hat min. $\frac{\lambda}{2} |V|$ I.simp.-Knoten
	\end{enumerate}
\end{KITinfoblock}
\ \\
Die Anzahl Friendly-Knoten in $G$ wird mit $n_f$ notiert.

\end{frame}


\begin{frame}
\frametitle{Algorithmus}
\framesubtitle{Fall: Min. $\lambda$ Friendly-Knoten}
\begin{enumerate}
	\item Finde Maximal Matching $M \subseteq E$
	\item Jede Kante in $M$ kontrahieren $\rightarrow \widetilde{G} = (\widetilde{V}, \widetilde{E})$ entsteht
	\item Kompletten Algorithmus auf $\widetilde{G}$ ausführen um Baumzerteilung $(Y,T)$ von $\widetilde{G}$ zu berechnen \\
		$\rightarrow$ Wenn Baumweite von $\widetilde{G} > k \Rightarrow$ \textcolor{red}{STOP} (\textcolor{cyan}{LEMMA 3.4.})
	\item Mit \textcolor{cyan}{LEMMA 3.3.} $(2k+1)$-Zerteilung $(X,T)$ von $G$ aus $(Y,T)$ erstellen %Weite 2k+1
	\item Mit \textcolor{cyan}{THEOREM 2.4.} prüfen ob Baumweite von $G > k$ ist, falls nein: \\
		$\rightarrow$ $k$-Baumzerteilung von $G$ ausgeben
\end{enumerate}
\end{frame}



\begin{frame}
\frametitle{Laufzeitanalyse}
\framesubtitle{Fall 1}

\begin{enumerate}
	\item[1.] Maximal Matching $M \subseteq E$ finden
\end{enumerate}
\ \\
\ \\
\ \\
Ein Maximal Matching kann greedy in $O(|V|+|E|)$ gefunden werden. \\
\ \\


\begin{KITinfoblock}{Lemma 2.3.}
	tw($G$) $\leq k \Rightarrow |E| \leq k|V| - \frac{1}{2} k (k+1)$
\end{KITinfoblock}
\ \\
\ \\
$\Rightarrow |E| \in O(|V|)$ \\
$\Rightarrow$ Max. Matching kann in $O(|V|)$ gefunden werden
\end{frame}


\begin{frame}
\frametitle{Laufzeitanalyse}
\framesubtitle{Fall 1}

\begin{enumerate}
	\item[2.] Jede Kante in $M$ kontrahieren um Graphen $\widetilde{G}$ zu erhalten
\end{enumerate}
\ \\
\ \\
Eine Kante kann in $O(1)$ kontrahiert werden, liegt der Graph als Adjazenzliste vor.
\ \\
\ \\
\begin{KITinfoblock}{Lemma 2.3.}
	tw($G$) $\leq k \Rightarrow |E| \leq k|V| - \frac{1}{2} k (k+1)$
\end{KITinfoblock}
\ \\
\ \\
$|M| \subseteq E$ und $E \in O(|V|)$ \\
$\Rightarrow$ Alle Kanten können in $O(|V|)$ kontrahiert werden.
\end{frame}


\begin{frame} %TODO Verstehen
\frametitle{Laufzeitanalyse}
\framesubtitle{Fall 1}

\begin{enumerate}
	\item[3.] Kompletten Algorithmus auf $\widetilde{G}$ ausführen um Baumzerteilung $(Y,T)$ von $\widetilde{G}$ auszugeben
\end{enumerate}
\ \\
\ \\
Ein Maximal Matching hat min. $\frac{n_f}{2 d}$ Kanten. \\
\ \\
Für jeden Friendly-Knoten gilt:
\begin{itemize}
	\item Er ist Endpunkt von einem $m \in M$, oder
	\item Er ist adjazent zu einem Friendly-Knoten, der Endpunkt ist
\end{itemize}
$\Rightarrow \forall m \in M$ werden max. $2d$ Friendly-Knoten assoziiert, die Endpunkt sind oder adjazent zu einem Friendly-Endpunkt sind.\\
\ \\
Ist ein Friendly-Knoten nicht assoziiert so ist $M$ nicht maximal $\Rightarrow |M| \geq \frac{n_f}{2d}$
\end{frame}

\begin{frame} %TODO Verstehen
\frametitle{Laufzeitanalyse}
\framesubtitle{Fall 1}

\begin{enumerate}
	\item[3.] Kompletten Algorithmus auf $\widetilde{G}$ ausführen um Baumzerteilung $(Y,T)$ von $\widetilde{G}$ auszugeben
\end{enumerate}
\ \\
\ \\
Ein Maximum Matching hat min. $\frac{n_f}{2d}$ Kanten. \\
$\Rightarrow |\widetilde{V}| = (1 - \frac{1}{2d(4k^2+12k+16)}) \cdot |V|$
\end{frame}


\begin{frame}
\frametitle{Laufzeitanalyse}
\framesubtitle{Fall 1}

\begin{enumerate}
	\item[4.] Mit \textcolor{cyan}{LEMMA 3.3.} Zerteilung $(X,T)$ von $G$ aus $(Y,T)$ erstellen %Weite 2k+1
\end{enumerate}
\ \\
\ \\
$
f_M: V \mapsto \widetilde{V}:
\begin{cases} %TODO Umformatieren
	f_M(v) = v &\text{Wenn } v \text{ nicht Endpunkt in } M \text{ ist} \\
	f_M(v) = f_M(w) &\text{sonst}
\end{cases}$\\
Dabei ist $f_m(w)$ der Knoten der bei der Kontraktion von $(v,w) \in M$ bleibt. \\
\ \\
\ \\
Ist $(Y,T)$ Zerteilung von $\widetilde{G}$, so ist $(X,T)$ mit $X_i = \{ v \in V | f_M(v) \in Y_i \}$ Zerteilung von $G$ mit Weite max. $2k+1$
\end{frame}


\begin{frame}
\frametitle{Laufzeitanalyse}
\framesubtitle{Fall 1}

\begin{enumerate}
	\item[5.] prüfen ob Baumweite von $G > k$ ist $\Rightarrow$ \textcolor{red}{STOP}
	\item[5.1.] $k$-Zerteilung von G errechnen

\end{enumerate}
\ \\
\ \\
\ \\

\begin{KITinfoblock}{Lemma 2.4.}
$\forall k,l \in \mathbb{N}$ $\exists$ Linearzeitalgorithmus, welcher aus einem Graph $G=(V,E)$ und einer $l$-Zerteilung prüft ob die Baumweite von $G$ max. $k$ ist und eine $k$-Zerteilung errechnet
\end{KITinfoblock}

\ \\
\ \\
\ \\
Laufzeit: $O(l^{l-2} \cdot ((2l+3)^{2l+3} \cdot (\frac{8}{3} 2^{2k+2})^{2l+3})^{2l-1})) \in O(2^{k^3})$ bei $l \in O(k)$
\end{frame}

%%%%%%%%%%%%%%%%%%%%%%%%%%%%%%%%%%%%%%%%%%%%%%%%%%%%%%%%%%%%%%%%%%%%%%%%%%%%%%%%%%%%%


\begin{frame}


\begin{KITinfoblock}{Lemma 4.2.}
	$G$ hat Baumweite max. $k \Rightarrow$ Es gilt mindestens:
	\begin{enumerate}
		\item $G$ hat min. $\frac{|V|}{4k^2+12k+16} =: \lambda$ Friendly-Knoten, oder
		\item Der Improved-Graph von $G$ hat min. $\frac{\lambda}{2} |V|$ I.simp.-Knoten
	\end{enumerate}
\end{KITinfoblock}
\ \\
Die Anzahl Friendly-Knoten in $G$ wird mit $n_f$ notiert.

\end{frame}


\begin{frame}
\frametitle{Verbesserter Graph $G'$}
\framesubtitle{Erstellung und Eigenschaften}

$G'=(V,E')$ ist $G=(V,E)$ mit Kanten $(v,w) \in E' \forall v,w \in V$ sodass $v,w$ min. $k+1$ gem. Nachbarn mit Grad max. $k$ haben. \\
\ \\
\begin{KITinfoblock}{Lemma 4.1.}
	tw($G$)$\leq k \Leftrightarrow$ tw($G'$)$\leq k$. \\
	Jede $k$-Zerteilung von $G$ ist auch eine $k$-Zerteilung von $G'$ und umgekehrt.
\end{KITinfoblock}

\end{frame}


\begin{frame}
\frametitle{Algorithmus}
\framesubtitle{Fall: Max. $\lambda - 1$ Friendly-Knoten}
\begin{enumerate}
	\item Improved-Graph $G'$ berechnen \\
	$\rightarrow \exists$ I.simp.-Knoten $v$ mit deg($v$) = $k+1 \Rightarrow$ \textcolor{red}{STOP}
	
	\item Alle I.simp.-Knoten in Menge $SL$ und von $G$ entfernen $\Rightarrow \widehat{G}$ entsteht \\
	$\rightarrow |SL| < c_2 \cdot |V| \Rightarrow$ \textcolor{red}{STOP} (\textcolor{cyan}{THEOREM 4.2.})
	
	\item Algorithmus rekursiv auf $\widehat{G}$ ausführen $\Rightarrow$ Ausgabe von Zerteilung $(Y,T)$ von $\widehat{G}$ \\
	$\rightarrow$ tw($\widehat{G}$) $> k \Rightarrow$ \textcolor{red}{STOP} ($\widehat{G}$ Teilgraph von $G \Rightarrow$ tw($G$) $> k$)
	
	\item Füge $SL$ wieder in die Zerteilung $(Y,T)$ ein \\
	$\Rightarrow$ Baumzerteilung $(X,T)$ von $G$ mit Baumweite max. $k$
\end{enumerate}
\end{frame}



\begin{frame}
\frametitle{Laufzeitanalyse}
\framesubtitle{Fall 2}

\begin{enumerate}
	\item[1.] Den Improved-Graph $G'$ berechnen
	\item[1.1.] Alle I-simp.-Knoten von $G$ in Menge $SL$ zusammenfassen
\end{enumerate}
\ \\
\ \\

Wir definieren $Q = \{ ((v_i, v_j), - ) | (v_i, v_j) \in E, i < j \}$ \\
\hspace{2.86cm}$\cup \{ ((v_i, v_j), v) | v_i, v_j \in N_G(v), i < j \wedge deg(v) \leq k\}$ \\
\ \\
und $Q_{v_i,v_j} = \{ ((v_i, v_j), v) | v_i, v_j$ fest$, v \in V \} \subseteq Q$ \\



\end{frame}


\begin{frame}[t]
\frametitle{Laufzeitanalyse}
\framesubtitle{Fall 2}

\begin{columns}
	\column{.5\textwidth}
	\begin{block}{$Q$ für $k=2$:}
		\begin{itemize}
			\item $((1,3), -)$
			\item $((1,4), -)$
			\item $((1,5), -)$
			\item $((1,7), -)$
			\item $((2,4), -)$
			\item $((2,5), -)$
			\item $((2,6), -)$
			\item $((2,7), -)$
		\end{itemize}
	\end{block}
	
	\column{.5\textwidth}
	\includegraphics[width=0.7\textwidth]{images/Graph_Queue_1.png}
\end{columns}
\end{frame}

\begin{frame}[t]
\frametitle{Laufzeitanalyse}
\framesubtitle{Fall 2}

\begin{columns}
	\column{.5\textwidth}
	\begin{block}{$Q$:}
			\begin{itemize}
				\item $((1,3), -)$
				\item $((1,4), -)$
				\item $((1,5), -)$
				\item $((1,7), -)$
				\item $((2,4), -)$
				\item $((2,5), -)$
				\item $((2,6), -)$
				\item $((2,7), -)$
			
				\item $((1,2), 4)$
				\item $((1,2), 5)$
				\item $((1,2), 7)$
		\end{itemize}
	\end{block}

	\column{.5\textwidth}
		\includegraphics[width=0.7\textwidth]{images/Graph_Queue_1.png}
\end{columns}
\end{frame}

\begin{frame}[t]
\frametitle{Laufzeitanalyse}
\framesubtitle{Fall 2}

\begin{columns}
	\column{.5\textwidth}
	\begin{block}{$Q$, erster BucketSort:}
		\begin{itemize}
			\item $((1,3), -)$
			\item $((1,4), -)$
			\item $((1,5), -)$
			\item $((1,7), -)$
			\item $((1,2), 4)$
			\item $((1,2), 5)$
			\item $((1,2), 7)$
			\item $((2,4), -)$
			\item $((2,5), -)$
			\item $((2,6), -)$
			\item $((2,7), -)$
		\end{itemize}
	\end{block}
	
	\column{.5\textwidth}
	\includegraphics[width=0.7\textwidth]{images/Graph_Queue_1.png}
\end{columns}
\end{frame}


\begin{frame}[t]
\frametitle{Laufzeitanalyse}
\framesubtitle{Fall 2}

\begin{columns}
	\column{.5\textwidth}
	\begin{block}{$Q$, zweiter BucketSort:}
		\begin{itemize}
			\item $((1,2), 4)$
			\item $((1,2), 5)$
			\item $((1,2), 7)$
			\item $((1,3), -)$
			\item $((2,4), -)$
			\item $((1,4), -)$
			\item $((1,5), -)$
			\item $((2,5), -)$
			\item $((2,6), -)$
			\item $((1,7), -)$
			\item $((2,7), -)$
		\end{itemize}
	\end{block}
	
	\column{.5\textwidth}
	\includegraphics[width=0.7\textwidth]{images/Graph_Queue_1.png}
\end{columns}
\end{frame}


\begin{frame}
\frametitle{Laufzeitanalyse}
\framesubtitle{Fall 2}

$Q_{v_i,v_j} = \{ ((v_i, v_j), v) | v_i, v_j$ fest$, v \in V \} \subseteq Q$ \\
\ \\
Falls $|Q_{v_i, v_j}| \geq k \rightarrow (v_i, v_j) \in E'$, da $v_i$ und $v_j$ nun min. $(k+1)$ gemeinsame Nachbarn haben. \\
\ \\
Für jedes Element aus der oberen Menge und wenn $((v_i, v_j), -) \in Q$: \\
Füge $(v_i, v_j)$ für jedes $v \in V$ zu $Q_v$ hinzu, sodass $Q_v$ alle Kanten von Nachbarn von $v$ enthält.\\
\end{frame}

\begin{frame}[t]
\frametitle{Laufzeitanalyse}
\framesubtitle{Fall 2}

\begin{columns}
	\column{.5\textwidth}
	\begin{block}{$Q$, zweiter BucketSort:}
		\begin{itemize}
			\item[1] $((1,2), 4)$
			\item[2] $((1,2), 5)$
			\item[3] $((1,2), 7)$
			\item $((1,3), -)$
			\item $((2,4), -)$
		\end{itemize}
	\vdots
	\ \\
	\ \\
	$S$: \\
	\begin{tabular}{| c | c | c | c | c | c | c |}
		\hline
		1 & 2 & 3 & 4 & 5 & 6 & 7 \\
		\hline
		{\scriptsize leer} & {\scriptsize leer} & {\scriptsize leer} & {\scriptsize leer} & {\scriptsize leer} & {\scriptsize leer} & {\scriptsize leer} \\
		\hline
	\end{tabular}
	\end{block}
	
	\column{.5\textwidth}
	\includegraphics[width=0.7\textwidth]{images/Graph_Queue_1.png}
\end{columns}
\end{frame}


\begin{frame}
\frametitle{Laufzeitanalyse}
\framesubtitle{Fall 2}

Der Graph $G' = (V,E')$ kann also aus den verschiedenen $Q_{v_i,v_j}$ ausgelesen werden. \\
\ \\
Das finden von I-simp.-Knoten ist durch $Q_v$ nun auch möglich: Da alle Nachbarn von $v$ in $Q_v$ sind, kann schnell geprüft werden ob $N_{G'}(v)$ eine Clique formt.

\end{frame}


\begin{frame}
\frametitle{Laufzeitanalyse}
\framesubtitle{Fall 2}

Queue $Q$ für Menge $Q$, Array $S$ aus Listen für die $Q_v$'s. \\
\ \\
\ \\

\begin{enumerate}
	\item Knoten ordnen $(v_1, v_2 \dots v_n)$
	\item $\forall (v_i, v_j) \in E, i < j:$ Lege $((v_i, v_j), -)$ auf $Q$
	\item $\forall v \in V:$ Lege alle $((v_i, v_j), v)$ mit $v_i, v_j \in N_G(v)$ und $i < j$ auf $Q$
	\item Bucket-sortiere $Q$ zwei mal: Ein mal nach dem ersten, dann nach zweiten Eintrag
\end{enumerate}


\end{frame}


\begin{frame}
\frametitle{Laufzeitanalyse}
\framesubtitle{Fall 2}

Sind nach dem Sortieren von $Q$ $(k+1)$ Einträge für gleiches $(v_i, v_j)$ in $Q$ untereinander $\rightarrow (v_i, v_j) \in E'$ \\
\ \\
Ist für solche $v_i, v_j$ auch $((v_i, v_j), -)$ in $Q$: Füge für jedes $((v_i, v_j), v)$ das Tupel $(v_i, v_j)$ in alle $S[v]$ ein.\\
\ \\
Ist in $S[v]$ jedes $v_i$ mit jedem $v_j$ verbunden: $N_G'(v)$ bildet Clique $\Rightarrow v$ ist I-simp. $\Rightarrow v \in SL$
\end{frame}


\begin{frame}
\frametitle{Laufzeitanalyse}
\framesubtitle{Fall 2}

\begin{enumerate}
	\item[2.] $SL$ von $G$ entfernen $\rightarrow \widehat{G}=(\widehat{V}, \widehat{E})$ entsteht %TODO Formulierung "Entsteht"?
	\item[3.] Algorithmus rekursiv auf $\widehat{G}$ ausführen
\end{enumerate}
\ \\
\ \\
\ \\
\ \\
Graph $\widehat{G}$ hat nach Entfernung von $SL$ $(1 - c_2) \cdot |V|$ Knoten. \\
\ \\
Wie in Fall 1 sind alle rekursiven Aufrufe in $O(|V|)$ möglich. \\

\end{frame}


\begin{frame}
\frametitle{Laufzeitanalyse}
\framesubtitle{Fall 2}

\begin{enumerate}
	\item[4.] Füge $SL$ wieder in die Zerteilung $(Y,T)$ ein
\end{enumerate}
\ \\
\ \\
\begin{enumerate}
	\item $\forall v \in SL$: Finde ein $Y_{i_v} \in Y$ in dem alle Nachbarn von $v$ sind ($N_G(v) \subseteq Y_{i_v}$)
	\item Füge $Y_{j_v} = \{ \{v\} \cup N_G(v) \}$ zu $Y$ hinzu und mache es adjazent zu $Y_{i_v}$ \\
	$\Rightarrow$ Baumzerteilung von $G$ mit Baumweite max. $k$
\end{enumerate}
\ \\
$Y_{i_v}$ existiert für jedes $v$, da I-simp.-Knoten in $G$ nicht adjazent sind und $N_G(v)$ eine Clique formt. \\
\textcolor{cyan}{LEMMA 2.1.i)}: "$(X,T)$ Zerteilung von $G$ und $W \subseteq V$ formt Clique in $G \Rightarrow \exists i \in I: W \subseteq X_i$"
\end{frame}

\begin{frame}
\frametitle{Laufzeitanalyse}
\framesubtitle{Fall 2}

\begin{enumerate}
	\item[4.] Füge $SL$ wieder in die Zerteilung $(Y,T)$ ein
\end{enumerate}
\ \\
\ \\

\begin{itemize}
	\item $\forall l \leq k$: Nimm Queue $Q_l$ wo alle Paare $((v_{i_1}, v_{i_2}, \dots v_{i_l}), i)$ für $v_{i_x} \in Y_i$ und für alle $i \in I, i_1 < i_2 \dots i_l$ hinzugefügt werden
	\item Füge zu jedem $Q_l$ noch alle Paare $((v_{i_1}, v_{i_2}, \dots v_{i_l}), v)$ wo $v$ I-simp. ist und $N_G(v) = \{ (v_{i_1}, v_{i_2}, \dots v_{i_l} | i_1 < i_2 \dots i_l \}$
	\item Sortiere jedes $Q_l$ $l$-mal, einmal für jedes $v_{i_x}$ im Tupel. Es entsteht wieder eine Sortierung der $v_i$
	\item Für jedes Tupel $((\dots), v)$ kann nun schnell das Tupel $((\dots), i)$ gefunden werden
	\item Mache den neuen Knoten $j_v$ mit $X_{i_j} \{ v \} \cup N_G(v)$ zu dem $i$ adjazent und füge ihn zu $Y$ hinzu
\end{itemize}
\ \\
Diese Schritte sind alle zusammen in $O(n)$ ausführbar.
\end{frame}

\begin{frame}
\frametitle{Abschluss}
Sämtliche Operationen sind in $O(n)$ wenn $k$ Konstant ist. \\
\ \\
Ist $k$ nicht konstant, sondern als Variable Teil der Eingabe, so ist der Algorithmus nicht mehr linear und das Problem wird NP-schwer. \\
\ \\
Der konstante Faktor ist deutlich zu hoch für praktische Anwendung, selbst schon für $k=4$. \\
Allerdings wurde bei vielen Operationen grob geschätzt. Es ist zu erwarten, dass die Konstante noch sinken wird. \\
\end{frame}




\begin{frame}
\frametitle{Erkenntnisse}
Der Algorithmus ist Basis für zwei weitere Theoreme: \\
\begin{itemize}
	\item $\exists$ Linearzeit-Erkennungsalgorithmus für jede Klasse an Graphen die nicht alle planaren Graphen enthält und in ihren Minoren abgeschlossen ist
	\item $\forall k \in \mathbb{N}: \exists$ Linearzeitalgo der prüft ob $G=(V,E)$ Pfadweite max. $k$ hat und eine Pfad-Zerteilung ausgibt
\end{itemize}
\ \\
\ \\
Außerdem ist die Erkennung von I.-simp.-Knoten sehr effizient und kann gut als Grundlage für andere Algorithmen genutzt werden.
\end{frame}











\end{document}
