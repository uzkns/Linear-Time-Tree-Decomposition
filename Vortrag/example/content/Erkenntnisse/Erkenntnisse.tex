\begin{frame}
\frametitle{Erkenntnisse}

Der konstante Faktor $k^3$ ist deutlich zu hoch für praktische Anwendung., selbst schon für $k=4$ \\
Allerdings wurde bei vielen Operationen grob geschätzt. Es ist zu erwarten, dass die Konstante noch sinken wird. \\
\ \\
Allerdings ist der Algorithmus Basis für zwei weitere Theoreme: \\
\begin{itemize}
	\item $\exists$ Linearzeit-Erkennungsalgorithmus für jede Klasse an Graphen die nicht alle planaren Graphen enthält und in ihren Minoren abgeschlossen ist
	\item $\forall k \in matbb{N}$ $\exists$ Linearzeitalgo der rüft ob $G=(V,E)$ Pfadweite max. $k$ hat und eine Pfad-Zerteilung ausgibt
\end{itemize}
\ \\
\ \\
Ausserdem ist die Erkennung von I.-simp.-Knoten sehr effizient und kann gut als Grundlage für andere Algorithmen genutzt werden



\end{frame}