\begin{frame}
\frametitle{Abschluss}
Sämtliche Operationen sind in $O(n)$ wenn $k$ Konstant ist. \\
\ \\
Ist $k$ nicht konstant, sondern als Variable Teil der Eingabe, so ist der Algorithmus nicht mehr linear. (Schritt 5.1. ist in $O(k^3)$) \\
\ \\
Der konstante Faktor $k^3$ ist deutlich zu hoch für praktische Anwendung, selbst schon für $k=4$. \\
Allerdings wurde bei vielen Operationen grob geschätzt. Es ist zu erwarten, dass die Konstante noch sinken wird. \\
\end{frame}




\begin{frame}
\frametitle{Erkenntnisse}
Der Algorithmus ist Basis für zwei weitere Theoreme: \\
\begin{itemize}
	\item $\exists$ Linearzeit-Erkennungsalgorithmus für jede Klasse an Graphen die nicht alle planaren Graphen enthält und in ihren Minoren abgeschlossen ist
	\item $\forall k \in \mathbb{N}: \exists$ Linearzeitalgo der prüft ob $G=(V,E)$ Pfadweite max. $k$ hat und eine Pfad-Zerteilung ausgibt
\end{itemize}
\ \\
\ \\
Außerdem ist die Erkennung von I.-simp.-Knoten sehr effizient und kann gut als Grundlage für andere Algorithmen genutzt werden.
\end{frame}