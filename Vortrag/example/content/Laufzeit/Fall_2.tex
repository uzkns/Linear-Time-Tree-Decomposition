
\begin{frame}
\frametitle{Laufzeitanalyse}
\framesubtitle{Fall 2}

\begin{enumerate}
	\item[1.] Den Improved-Graph $G'$ berechnen
	\item[1.1.] Alle I-simp.-Knoten von $G$ in Menge $SL$ zusammenfassen
\end{enumerate}
\ \\
\ \\

Wir definieren $Q = \{ ((v_i, v_j), - ) | (v_i, v_j) \in E, i < j \}$ \\
\hspace{2.86cm}$\cup \{ ((v_i, v_j), v) | v_i, v_j \in N_G(v), i < j \wedge deg(v) \leq k\}$ \\
\ \\
und $Q_{v_i,v_j} = \{ ((v_i, v_j), v) | v_i, v_j$ fest$, v \in V \} \subseteq Q$ \\

\end{frame}


\begin{frame}
\frametitle{Laufzeitanalyse}
\framesubtitle{Fall 2}
$Q$:
\begin{itemize}
	\item $((1,3), -)$
	\item $((1,4), -)$
	\item $((1,5), -)$
	\item $((1,7), -)$
	\item $((2,4), -)$
	\item $((2,5), -)$
	\item $((2,6), -)$
	\item $((2,7), -)$
\end{itemize}
\end{frame}

\begin{frame}
\frametitle{Laufzeitanalyse}
\framesubtitle{Fall 2}
\begin{minipage}{0.5\textwidth}
	$Q$:
	\begin{itemize}
		\item $((1,3), -)$
		\item $((1,4), -)$
		\item $((1,5), -)$
		\item $((1,7), -)$
		\item $((2,4), -)$
		\item $((2,5), -)$
		\item $((2,6), -)$
		\item $((2,7), -)$
		
		\item $((1,2), 4)$
		\item $((1,2), 5)$
		\item $((1,2), 7)$
	\end{itemize}
\end{minipage}

\begin{minipage}{0.5\textwidth}
	
\end{minipage}
\end{frame}


\begin{frame}
\frametitle{Laufzeitanalyse}
\framesubtitle{Fall 2}

$Q_{v_i,v_j} = \{ ((v_i, v_j), v) | v_i, v_j$ fest$, v \in V \} \subseteq Q$ \\
\ \\
Falls $|Q_{v_i, v_j}| \geq k \rightarrow (v_i, v_j) \in E'$, da $v_i$ und $v_j$ nun min. $(k+1)$ gemeinsame Nachbarn haben. \\
\ \\
Für jedes Element aus der oberen Menge und wenn $((v_i, v_j), -) \in Q$: \\
Füge $(v_i, v_j)$ für jedes $v \in V $zu $Q_v$ hinzu, sodass $Q_v$ alle Kanten von Nachbarn von $v$ enthält.\\
\end{frame}


\begin{frame}
\frametitle{Laufzeitanalyse}
\framesubtitle{Fall 2}

Der Graph $G' = (V,E')$ kann also aus den verschiedenen $Q_{v_i,v_j}$ ausgelesen werden. \\
\ \\
Das finden von I-simp.-Knoten ist durch $Q_v$ nun auch möglich: Da alle Nachbarn von $v$ in $Q_v$ sind, kann schnell geprüft werden ob $N_{G'}(v)$ eine Clique formt.

\end{frame}


\begin{frame}
\frametitle{Laufzeitanalyse}
\framesubtitle{Fall 2}

Queue $Q$ für Menge $Q$, Array $S$ aus Listen für die $Q_v$'s. \\
\ \\
\ \\

\begin{enumerate}
	\item Knoten ordnen $(v_1, v_2 \dots v_n)$
	\item $\forall (v_i, v_j) \in E, i < j:$ Lege $((v_i, v_j), -)$ auf $Q$
	\item $\forall v \in V:$ Lege alle $((v_i, v_j), v)$ mit $v_i, v_j \in N_G(v)$ und $i < j$ auf $Q$
	\item Bucket-sortiere $Q$ zwei mal: Ein mal nach dem ersten, dann nach zweiten Eintrag
\end{enumerate}


\end{frame}


\begin{frame}
\frametitle{Laufzeitanalyse}
\framesubtitle{Fall 2}

Sind nach dem Sortieren von $Q$ $(k+1)$ Einträge für gleiches $(v_i, v_j)$ in $Q$ untereinander $\rightarrow (v_i, v_j) \in E'$ \\
\ \\
Ist für solche $v_i, v_j$ auch $((v_i, v_j), -)$ in $Q$: Füge für jedes $((v_i, v_j), v)$ das Tupel $(v_i, v_j)$ in alle $S[v]$ ein.\\
\ \\
Ist in $S[v]$ jedes $v_i$ mit jedem $v_j$ verbunden: $N_G'(v)$ bildet Clique $\Rightarrow v$ ist I-simp. $\Rightarrow v \in SL$
\end{frame}


\begin{frame}
\frametitle{Laufzeitanalyse}
\framesubtitle{Fall 2}

\begin{enumerate}
	\item[2.] $SL$ von $G$ entfernen $\rightarrow \widehat{G}=(\widehat{V}, \widehat{E})$ entsteht %TODO Formulierung "Entsteht"?
	\item[3.] Algorithmus rekursiv auf $\widehat{G}$ ausführen
\end{enumerate}
\ \\
\ \\
\ \\
\ \\
Graph $\widehat{G}$ hat nach Entfernung von $SL$ $(1 - c_2) \cdot |V|$ Knoten. \\
\ \\
Wie in Fall 1 sind alle rekursiven Aufrufe in $O(|V|)$ möglich. \\

\end{frame}


\begin{frame}
\frametitle{Laufzeitanalyse}
\framesubtitle{Fall 2}

\begin{enumerate}
	\item[4.] Füge $SL$ wieder in die Zerteilung $(Y,T)$ ein
\end{enumerate}
\ \\
\ \\
\begin{enumerate}
	\item $\forall v \in SL$: Finde ein $Y_{i_v} \in Y$ in dem alle Nachbarn von $v$ sind ($N_G(v) \subseteq Y_{i_v}$)
	\item Füge $Y_{j_v} = \{ \{v\} \cup N_G(v) \}$ zu $Y$ hinzu und mache es adjazent zu $Y_{i_v}$ \\
	$\Rightarrow$ Baumzerteilung von $G$ mit Baumweite max. $k$
\end{enumerate}
\ \\
$Y_{i_v}$ existiert für jedes $v$, da I.simp.-Knoten in $G$ nicht adjazent sind und $N_G(v)$ eine Clique formt. \\
\textcolor{cyan}{LEMMA 2.1.i)}: "$(X,T)$ Zerteilung von $G$ und $W \subseteq V$ formt Clique in $G \Rightarrow \exists i \in I: W \subseteq X_i$"
\end{frame}

\begin{frame}
\frametitle{Laufzeitanalyse}
\framesubtitle{Fall 2}

\begin{enumerate}
	\item[4.] Füge $SL$ wieder in die Zerteilung $(Y,T)$ ein
\end{enumerate}
\ \\
\ \\

\begin{itemize}
	\item $\forall l \leq k$: Nimm Queue $Q_l$ wo alle Paare $((v_{i_1}, v_{i_2}, \dots v_{i_l}), i)$ für $v_{i_x} \in Y_i$ und für alle $i \in I, i_1 < i_2 \dots i_l$ hinzugefügt werden
	\item Füge zu jedem $Q_l$ noch alle Paare $((v_{i_1}, v_{i_2}, \dots v_{i_l}), v)$ wo $v$ I-simp. ist und $N_G(v) = \{ (v_{i_1}, v_{i_2}, \dots v_{i_l} | i_1 < i_2 \dots i_l \}$
	\item Sortiere jedes $Q_l$ $l$-mal, einmal für jedes $v_{i_x}$ im Tupel. Es entsteht wieder eine Sortierung der $v_i$
	\item Für jedes Tupel $((\dots), v)$ kann nun schnell das Tupel $((\dots), i)$ gefunden werden
	\item Mache den neuen Knoten $j_v$ mit $X_{i_j} \{ v \} \cup N_G(v)$ zu dem $i$ adjazent und füge ihn zu $Y$ hinzu
\end{itemize}
\ \\
Diese Schritte sind alle zusammen in $O(n)$ ausführbar.
\end{frame}
